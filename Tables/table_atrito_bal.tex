\begin{table}[H]

\caption{\label{tab:atrito}Análise de atrito. Balanceamento de variáveis selecionadas}
\centering
\begin{threeparttable}
\begin{tabular}[t]{lrrrrrr}
\toprule
Tratamento & Gênero & Idade & Renda & População & Dens. pop. & Compliance\\
\midrule
T0 & 0.6458 & 48.0170 & 20928.4068 & 45815.2715 & 8.1711 & 0.9355\\
T1 & 0.6403 & 47.7868 & 21100.3921 & 52084.9822 & 7.6001 & 0.9322\\
T2 & 0.6211 & 47.7127 & 21106.0117 & 48882.0302 & 6.5860 & 0.9337\\
T3 & 0.6138 & 47.8580 & 21077.8894 & 51027.8338 & 6.6317 & 0.9313\\
T4 & 0.6240 & 47.8056 & 20945.2352 & 48251.5259 & 6.5957 & 0.9318\\
\addlinespace
T5 & 0.6177 & 47.7952 & 20864.3756 & 43273.7019 & 6.3919 & 0.9308\\
T6 & 0.6320 & 47.8117 & 20966.9995 & 46539.3467 & 6.4614 & 0.9324\\
Anova p-valor & 0.4319 & 0.0000 & 0.0095 & 0.0936 & 0.0094 & 0.1122\\
\bottomrule
\end{tabular}
\begin{tablenotes}
\item \textit{Nota:} 
\item Gênero igual a zero para mulher. Demais variáveis são denominadas em nível municipal, por exemplo Idade refere-se a idade média dos habitantes do município de residência do indivíduo.
\end{tablenotes}
\end{threeparttable}
\end{table}
