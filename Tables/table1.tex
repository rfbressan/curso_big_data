\begin{table}[H]

\caption{\label{tab:tab1}Balanceamento de características individuais e por município por tipo de tratamento.}
\centering
\fontsize{10}{12}\selectfont
\begin{threeparttable}
\begin{tabular}[t]{llrrrrrr}
\toprule
Tratamento & Descrição & Gênero & Idade & Renda & População & Dens. pop. & Compliance\\
\midrule
T0 & Sem Correio & 0.6458 & 48.0170 & 20928.4068 & 45815.2715 & 8.1711 & 0.9355\\
T1 & Correio & 0.6338 & 47.9969 & 20878.9958 & 43377.1935 & 8.5625 & 0.9352\\
T2 & Ameaça & 0.6367 & 47.9931 & 20901.1614 & 44542.5883 & 7.9605 & 0.9346\\
T3 & Info & 0.6260 & 48.0300 & 20882.6636 & 43903.0189 & 8.1142 & 0.9347\\
T4 & Info\&Ameaça & 0.6335 & 48.0051 & 20879.6138 & 43319.4736 & 8.3540 & 0.9352\\
\addlinespace
T5 & Moral & 0.6251 & 47.9982 & 20888.4584 & 44301.3718 & 8.4832 & 0.9343\\
T6 & Moral\&Ameaça & 0.6422 & 47.9904 & 20876.3062 & 43610.1972 & 8.0468 & 0.9343\\
Anova: & p-values & 0.1715 & 0.3993 & 0.9393 & 0.7577 & 0.5795 & 0.8614\\
\bottomrule
\end{tabular}
\begin{tablenotes}
\item \textit{Nota:} 
\item Gênero igual a zero para mulher. Demais variáveis são denominadas em nível municipal, por exemplo Idade refere-se a idade média dos habitantes do município de residência do indivíduo.
\end{tablenotes}
\end{threeparttable}
\end{table}
