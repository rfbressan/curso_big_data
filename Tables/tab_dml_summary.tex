\begin{table}[ht]
\centering
\caption{Efeitos heterogêneos do tratamento estimados por Double Machine Learning.}
\label{tab:dml-summary}
\begin{threeparttable}
\begin{tabular}{lllll}
\toprule
    X &    Correio &     Ameaça &       Info &      Moral \\
\midrule
 mean &  0.0680*** &  0.0810*** &  0.0620*** &  0.0690*** \\
  &   (0.0110) &   (0.0110) &   (0.0090) &   (0.0110) \\
  min &  0.0780*** &  0.0790*** &  0.1150*** &  0.0790*** \\
   &   (0.0140) &   (0.0210) &   (0.0320) &   (0.0190) \\
  25\% &  0.0700*** &  0.0690*** &  0.0590*** &  0.0650*** \\
   &   (0.0120) &   (0.0100) &   (0.0080) &   (0.0140) \\
  50\% &  0.0670*** &  0.0800*** &  0.0570*** &  0.0650*** \\
   &   (0.0090) &   (0.0130) &   (0.0110) &   (0.0110) \\
  75\% &  0.0840*** &  0.1120*** &  0.0780*** &  0.0770*** \\
   &   (0.0110) &   (0.0210) &   (0.0130) &   (0.0130) \\
  max &  0.1010*** &  0.1090*** &  0.0840*** &  0.0820*** \\
   &   (0.0190) &   (0.0240) &   (0.0150) &   (0.0180) \\
\bottomrule
\end{tabular}
\begin{tablenotes}
\item \emph{Nota: Os estágios de previsão foram floresta aleatória para $E[Y|\bfx]$ e regressão logística para $E[T|\bfx]$. O modelo final para o efeito condicional do tratamento, $\theta(\bfx)$, é uma floresta aleatória.}\\
\end{tablenotes}
\end{threeparttable}
\end{table}
