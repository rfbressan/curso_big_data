\begin{table}[H]

\caption{\label{tab:missings}Dados faltantes na amostra.}
\centering
\begin{threeparttable}
\begin{tabular}[t]{lrr}
\toprule
Variável & No. Faltantes & Completude\\
\midrule
deregistration & 48207 & 0.05\\
delivered & 2586 & 0.95\\
evasion\_1 & 9491 & 0.81\\
evasion\_2 & 9491 & 0.81\\
threat\_evasion\_D1 & 9491 & 0.81\\
\addlinespace
appeal\_evasion\_D1 & 9491 & 0.81\\
info\_evasion\_D1 & 9491 & 0.81\\
threat\_evasion\_D2 & 9491 & 0.81\\
appeal\_evasion\_D2 & 9491 & 0.81\\
info\_evasion\_D2 & 9491 & 0.81\\
\addlinespace
age & 32466 & 0.36\\
coverage & 9491 & 0.81\\
\bottomrule
\end{tabular}
\begin{tablenotes}
\item \textit{Nota:} 
\item Completude refere-se a proporção de linhas preenchidas contra faltantes, e varia de zero a um.
\end{tablenotes}
\end{threeparttable}
\end{table}
