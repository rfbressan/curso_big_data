\mode<presentation> {
  \usetheme[block=fill]{metropolis}
}

%%% FGV colors for metropolis
% As cores vou deixar definido, estão certas as
% cores da FGV. Só saber como usar agora
\definecolor{brunoblue}{HTML}{182936}
\definecolor{brunomarroon}{HTML}{704730}
\definecolor{brunolightgray}{HTML}{E6E6E6}
\definecolor{fgvblue}{HTML}{1C2F67}
\definecolor{fgvlightblue}{HTML}{0096D6}

% Structure
% \setbeamercolor{structure}{fg=fgvblue}
\setbeamercolor{normal text}{fg=fgvblue}
% \setbeamercolor{separator}{fg=brunomarroon, bg=brunomarroon}
% \setbeamercolor{footline}{bg=fgvlightblue}

% % Blocs
% \setbeamercolor{block body}{bg=brunolightgray}
% \setbeamercolor{block title}{bg=fgvblue, fg=white} 

% Math font
\usepackage{unicode-math}
\usepackage{FiraSans}
\setmainfont{Fira Sans Light}
% \setmathfont{STIX Two Math}

% \usetheme{Bruno}
% \background{fig/FGV-Logo_small.png}
\usepackage{booktabs}
\usepackage{threeparttablex}

% novos comandos
\newcommand{\bfx}{\mathbf{X}}
\newcommand{\bfw}{\mathbf{W}}
\DeclareMathOperator*{\argmin}{argmin}
